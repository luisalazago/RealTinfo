%%***( Conclusiones )***************************************************************

\begin{comment}
Este ítem hace explícita la importancia de reflexionar sobre el proceso de diseño (qué sucedió sin contratiempos, qué fue sorpresivo, qué podría ser mejor). \\

Aquí es importante que se consignen las reflexiones y lecciones aprendidas no sólo desde lo disciplinar, sino también sobre el posible impacto que podría tener la Ingeniería de Sistemas y Computación (desde la perspectiva del proyecto desarrollado este semestre) sobre la sociedad. \\

Debe contener la reflexión de cada uno de los miembros del equipo y también debe haber al menos una reflexión grupal consensuada adicional.
\end{comment}
Conclusión Dilan: Parcialmente, pese a las complicaciones se evidencia un avance en el desarrollo del proyecto. Ha sido importante comprender que la implementación de un sistema no solamente se puede ver desde la observación computacional o el funcionamiento interno del sistema sino que debe ser detallado desde cada aspecto por más mínimo que sea. Como por ejemplo, en su momento no se había tenido en cuenta el funcionamiento básico de los dispositivos, y aunque es cierto que estos pueden presentar fallas y batería baja en algún momento del transcurso del sistema, es por eso que es necesario ahorrar batería en su mayor posibilidad, por diferentes motivos, más allá de los económicos, que también juegan un papel importante. Aún quedan cosas por refinar y mejorar para la presentación final de sistema, espero y confío en nosotros como equipo que vamos a poder sacar adelante lo que resta y poder entregar un sistema funcional y eficiente.
\newline

Conclusión Luis: Dentro de toda la realización del proyecto, a medida que se desglosa en partes más específicas, comienzan a surgir más complicaciones, ya que se tienen que definir más cosas y hay que tener en cuenta más cosas a la hora de hacer los diagramas. A comparación de los diagramas MSC, el diagrama SDL representa la arquitectura del sistema más específica sobre como se comunicaría todo el sistema, sin embargo, por cuestiones de tiempo no se pudo realizar las ejecuciones del plan de pruebas y no se pudo implementar de manera correcta el diagrama SDL, ya que tiene errores al momento de ejecutarlo. Por lo cual, esto problemas serán solucionados para la siguientes entregas y se presentarán como un escalamiento de los diagramas, ya que serían la versión completa y terminada de toda la idea del sistema.
\newline

Conclusión Guido: El proyecto ha tenido un desarrollo relativamente lento, debido a las complicaciones presentadas a lo largo del desarrollo del mismo. Sin embargo, cabe destacar que el valor que tiene el proyecto para el entendimiento en los patrones de los animales dentro de su propio hábitat hacen que los esfuerzos valgan la pena. Debido a esto, se ha tenido que hacer un análisis extenso sobre la arquitectura del sistema para poder garantizar un funcionamiento óptimo y sobre todo garantizar un sistema libre de fallos que permita en todo momento su uso de manera eficiente en temas de costos y mantenimiento. Es en esta misma fase de análisis, donde todavía quedan muchas cosas para mejorar para lograr los objetivos propuestos y entregar un sistema funcional que permita generar valor dentro de los esfuerzos de la preservación animal.
\newline

Conclusión General: El proyecto de sensar la población de animales es prometedor y tiene un gran potencial para mejorar el entendimiento de la vida salvaje en temas de comportamiento y sobre todo los esfuerzos de conservación. El proyecto sigue en desarrollo pero contempla un futuro prometedor, que junto al uso de tecnología de IoT permite la captura de datos en tiempo real para ofrecer un alto valor a la hora de entender los hábitos, patrones de migración y uso del hábitat de los animales. Dicha información podrá ser usado para informar y ayudar en los esfuerzos de conservación animal con el objetivo de la protección de los mismo. Mientras hay todavía mucho trabajo por realizar, el potencial del proyecto hace que el esfuerzo sea valioso y enriquecedor.