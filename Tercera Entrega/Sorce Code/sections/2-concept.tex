%%***( Concepto )*******************************************************************

%%==================================================================================
\section{Definición del problema}

Actualmente los biólogos presentan un gran problema para la recolección de datos sobre animales, ya que en muchos casos su presencia puede afectar el resultado obtenido o puede ser impreciso por las condiciones que pueden presentar la toma de los datos.  \\

Con este proyecto se busca facilitar a los biólogos la recolección de los datos, ya que por medio de algunos dispositivos se espera recolectar los datos esperados sobre cualquier problemática. La idea es que exista un equipo de trabajo que pueda llevar control sobre el funcionamiento de los diferentes dispositivos para que el sistema se encuentre correctamente en todo momento, ya que para la recolección de los datos esto es algo fundamental. \\

Más allá de la facilidad que se tendría para la recolección de los datos, como se mencionaba en un inicio, la presencia del ser humano en la toma de los datos puede muchas veces puede alterar el resultado, ya que podría ser un factor adicional en un ambiente con animales que no debería estar. De igual forma, podría asegurar que en para la recolección de algunos datos sobre los cuales algunas veces la vida del animal corre peligro, es muy importante la inmediatez del reconocimiento de los datos. Por ejemplo, existe un problema actualmente con los murciélagos y es que son animales muy sensibles de tratar, para la investigación sobre estos, el método que se utiliza para poder capturarlos es el siguiente: se utiliza una red de Nylon que se conoce como "Red de Niebla", que es lo suficientemente delgada para que cuando caigan sobre ella no los hiera, sin embargo, puede suceder que si pasan mucho tiempo atrapados en la red pueden perder la vida. Es por esto que la intervención rápida de los biólogos o investigadores para recogerlo es muy importante. El método actual que se utiliza según las preguntas realizadas a los biólogos de la Universidad Javeriana de Cali, es el mismo mencionado anteriormente, sólo que, como la presencia de seres humanos altera el resultado, lo que se hace es montar la red de Niebla y se hace un registro cada 30 minutos por algún biólogo o investigador si la red ha capturado algo. 
Los biólogos nos reportaron que en muchas ocasiones no se puede obtener con exactitud el tiempo que llevan atrapados en la red, y algunos murciélagos por desgracia pierden la vida. Es por eso, que poder tener un sistema que detecte cuando un murciélago es atrapado podría ayudar con la exactitud de los datos tomados y salvar la vida de los murciélagos que son capturados para investigaciones.

\begin{itemize}
    \item ¿Qué y quiénes están alrededor de la problemática del proyecto?
    \item ¿Quiénes van a querer usar la solución de este problema?
    \item ¿Por qué la van a querer usar?
    \item ¿Qué necesidades generales existen en el área de trabajo asignada?
    \item En general, ¿Cuál es el contexto del problema?
\end{itemize}

%%==================================================================================
\section{Restricciones del contexto}

Actualmente, el proyecto está propuesto para diferentes localizaciones a nivel mundial ya que se puede buscar recolectar la misma información para diferentes animales. Sin embargo, pueden existir algunos problemas para el desarrollo de la recolección de los datos que pueden ser los siguientes:

\begin{itemize}
    \item Algunos países presentan grupos armados al margen de la ley que tienen custodiadas algunas zonas que contienen a su vez hábitat naturales de animales.
    \item Si bien es cierto que actualmente existen software que puedan reconocer animales que está enfocando una vídeo cámara, es importante tener en cuenta que la precisión de estos sistemas puede variar dependiendo de varios factores, como la calidad de la imagen o vídeo, la variabilidad de las características de los animales, y la complejidad del entorno en el que se encuentran.
    \item El proyecto requiere de unas contribuciones económicas de parte de las entidades que quisieran hacer parte, es por esto, que no llegar al presupuesto mínimo podría concluir a la no ejecución del proyecto. No obstante, se debe tener en cuenta, que más allá del precio de los dispositivos a utilizar, la consideración importante sobre ellos es que sean resistentes a los diferentes climas que pueden afrontar.
    \item Aunque este proyecto está enfocado en la recolección de datos sobre animales para el conocimiento sobre estos, es posible que algunas políticas sobre el cuidado de la biodiversidad en algunos países pueda que no permitan este sistema.
\end{itemize}

%%==================================================================================

\begin{comment}
    \section{Descripción de los usuarios potenciales}
    Definir al usuario o conjunto de usuarios sobre los que se identificaron las necesidades que motivaron el desarrollo del proyecto. \\
    
    Estas pueden ser algunas preguntas orientadoras:
    
    \begin{itemize}
        \item ¿Quién es, cómo es, dónde vive, cómo vive, cuál es su perfil familiar, laboral, motivacional, emocional, etc.?
        \item ¿Cuál es la historia que acompaña a los usuarios objetivo? 
        \item ¿Existen conexiones con otros actores dentro del ambiente que rodea al usuario?
    \end{itemize}
    
    Los usuarios potenciales del sistema podrían incluir cualquier persona interesada en conocer más sobre la vida silvestre y la ecología de su área local o de otras regiones. También podrían incluir estudiantes y educadores que utilizan los datos recolectados para enseñar y aprender sobre la vida silvestre y la conservación. 
\end{comment}

%%==================================================================================

\section{Identificación de necesidades de los usuarios objetivo}

\begin{itemize}
    \item Investigadores y biólogos que estudian la ecología y el comportamiento de los animales.
    \item Organizaciones de conservación que monitorean la salud y la población de especies en peligro de extinción.
    \item Agencias gubernamentales que regulan la caza, la pesca y otras actividades relacionadas con la vida silvestre.
    \item Empresas que trabajan en la industria de la energía, la minería y otras industrias que pueden afectar los hábitats de los animales.
\end{itemize}

Como fue mencionado en la definición del problema, los biólogos de la universidad manifestaron que para recolectar datos sobre animales en muchas ocasiones la presencia humana alteraba el posible resultado, es por esto que implementar dispositivos que puedan recolectar datos podría ayudar mucho. De igual forma, poder realizar monitoreos a distancia podría beneficiar la analítica de datos. \\
También, como se mencionó en la definición del problema, para capturar algunos animales sobre los que se quiere hacer investigaciones, es importante que la captura del animal se haga de la mejor manera y esta no arriesgue la vida del animal.
 %%==================================================================================
\section{Requerimientos y especificaciones de la solución propuesta}

\begin{itemize}
    \item Capacidad de almacenamiento: El sistema debe ser capaz de almacenar grandes cantidades de datos de vídeo y audio de alta calidad de forma eficiente.
    \item Duración de la batería: Los dispositivos de recolección de datos deben tener una duración de batería suficiente para funcionar durante largos periodos de tiempo.
    \item Resistencia a las condiciones ambientales: Los dispositivos deben ser capaces de soportar condiciones climáticas diversas, como lluvia, viento, temperaturas altas, entre otros.
    \item Precisión del reconocimiento de animales: El software de reconocimiento de animales debe tener una alta tasa de precisión en la identificación de especies y en el seguimiento del movimiento y comportamiento de animales.
    \item Fácil instalación y mantenimiento: Los dispositivos de recolección de datos deben ser fáciles de instalar y mantener, con un mínimo de piezas móviles y componentes que requieran mantenimiento.
    \item Protección de los datos: Los datos recolectados deben ser almacenados y transmitidos de forma segura y protegida para evitar cualquier acceso no autorizado o robo de información.
    \item Minimización de gastos energéticos y económicos: Si bien es cierto que se requieren dispositivos de buena calidad, se debe priorizar que el proyecto pueda ser llevado a cabo en un sistema que es óptimo en el ahorro de recursos. Es por esto que se necesita de un controlador que envíe una señal al dispositivo para que este se encienda y así poder ahorrar mayor batería en el transcurso de la ejecución del proyecto y así también, ahorrar gastos.
\end{itemize}
